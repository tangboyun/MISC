%%% permFDR.tex --- 
%% Version: $Id: permFDR.tex,v 0.0 2012/07/30 06:32:55 tangboyun Exp$
%% Copyright : (c) 2012 Boyun Tang
%% License : BSD-style
\documentclass{article}
\usepackage{algorithm}
\usepackage{algorithmic}
\usepackage{times}
\renewcommand{\vec}[1]{\mbox{\boldmath$#1$}}
\DeclareMathAlphabet{\mathsfsl}{OT1}{cmss}{m}{sl}
\newcommand{\tensor}[1]{\mathsfsl{#1}}

\begin{document}

%\section{permutation plug-in method for FDR calculation}
%\label{sec-fdr}
\begin{algorithm}[H]
\caption{Permutation Plug-in Method}
\label{perm-plug-in}
\algsetup{indent=2em}
\algsetup{linenodelimiter=\ }
\begin{algorithmic}[1]
\REQUIRE Expression dataset $\tensor{X}$ ($n$ samples with $p$ genes for each sample), a $k$-gene signature $S$, the weight $\vec{w}$ for each gene.
\STATE Compute the distance statistics $d_1,\ldots,d_n$ based on the signature $S$ using neareast template prediction (NTP) algorithm.
\STATE Randomly pick up $k$ gene $B$ times. In the $b$th permutation, compute statistics $d_1^b,\ldots,d_n^b$ based on the permuted signature.
      
\STATE For a range of values of the cutpoint $C$, compute $\hat{V} = \frac{1}{B}\sum_{j=1}^n\sum_{b=1}^BI_{(|d_j^b| >C)}$, and $\hat{R} = \sum_{j=1}^nI_{(|d_j|>C)}$
\STATE Estimate the FDR at the cutpoint $C$ by $\hat{FDR_C}=\hat{\pi_0}\frac{\hat{V}}{\hat{R}}$, where $\hat{\pi_0}=\frac{2}{n}\sum_{j=1}^nI_{(|d_j|\ge q)}$ and $q$ the median of all permuted values $|d_j^b|$, $j = 1,\ldots,n$,$b=1,\ldots,B$.
\end{algorithmic}
\end{algorithm}


% \begin{algorithm}[H]
%   \caption{Nearest Template Prediction}\label{ntp}
%   \algsetup{indent=2em}
%   \algsetup{linenodelimiter=\ }

%   \begin{algorithmic}[1]
%     \REQUIRE
%     \STATE
%   \end{algorithmic}

% \end{algorithm}

  
\end{document}


%%% Local Variables: 
%%% TeX-master: t
%%% End: 
